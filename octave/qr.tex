\documentclass{ximera}
\input{../preamble.tex}

\title{$QR$-factorization} \license{CC BY-NC-SA 4.0}
\begin{document}
\begin{abstract}
\end{abstract}
\maketitle
\section*{$QR$-factorization}

In this tutorial we will use Octave to perform $QR$-factorization. You can access our Octave code through the link at the bottom of each template.  Feel free to modify the code and experiment to learn more!  Alternatively, go to the \href{https://sagecell.sagemath.org/}{Sage Math Cell Webpage}, copy the code below into the cell, select OCTAVE as the language, and press EVALUATE.  

Recall the definition of $QR$-factorization.
\begin{definition}[\ref{def:QR-factorization}]
Let $A$ be an $m \times n$ matrix with independent columns. A \dfn{QR-factorization} of $A$ expresses it as $A = QR$ where $Q$ is $m \times n$ with orthonormal columns and $R$ is an invertible and upper triangular matrix with positive diagonal entries.
\end{definition}

\begin{template}\label{temp:qr}

\begin{verbatim}
% Find the QR-factorization of A
[Q,R]=qr(A)
\end{verbatim}
\end{template}

\begin{example}\label{ex:qrOrthonormalQ}
Find the $QR$-factorization of 
$$A=\begin{bmatrix}2 & -1 & 4 & 0\\
    3 & 1 & 1 & -2\\
    1 & 5 & -1 & 6\\
    -3 & 4 & 0 & 7\end{bmatrix}$$
and demonstrate that matrix $Q$ is orthogonal (columns are orthonormal).     
\begin{explanation}
    \begin{verbatim}
% Define A
A=[2 -1 4 0;
    3 1 1 -2;
    1 5 -1 6;
    -3 4 0 7];

% Find the QR-factorization of A
[Q,R]=qr(A)

% Verify that the columns of Q are unit vectors
for i=1:4
    normOfCol=norm(Q(:,i))
end    

% Verify that the columns of Q are pairwise orthogonal
for i=1:3
    counter=i+1;
        while counter <= 4
            dotProduct=dot(Q(:,i),Q(:,counter))
            counter=counter+1;
        end
end
    \end{verbatim}
  
\href{https://sagecell.sagemath.org/?z=eJyNT8FOwzAMvVfqP_gyqROtRNYC0iCHCsQVugOXaYeodailLoY0ZYKvJ1kzDW57PtiW_d6zF_CEmgxCnSa13K6gEFDB9X2agEcJwkexiq2AmzC_jW1RhlW42_k-TRbwTKYD1yM0m0Kr1rGlH-WIDbAO-tsm3-zkp83q5Ux4Q0v621OUO_JaHqa9GcN6A8oiTIYcfGGQGtNEswWSYl3N9obt_kU_8iBDlTXZOqelV0Z_hceFFh-K7IFGBLau53c2ajg7lbNTy5NxaCVdifh6wKGnAU8zeJBQnWcBHbtXy93UOunLeF8eUuSEY_8STjYx_zPzTx0_-wUZ_XQT&lang=octave&interacts=eJyLjgUAARUAuQ==}{Link to code}.
\end{explanation}  
In this example we used a $4\times 4$ square matrix.  Can you think of a way to modify the code for a generic $m\times n$ matrix?
\end{example}

Recall the following algorithm for using $QR$-factorization to approximate eigenvalues.

\begin{algorithm}[\ref{alg:qrEig}]
Let $A$ be an invertible matrix.

\emph{Step 1}: Define $A_{1} = A$ and factor it as $A_{1} = Q_{1}R_{1}$.

\emph{Step 2}: Define $A_{2} = R_{1}Q_{1}$ and factor it as $A_{2} = Q_{2}R_{2}$.

\emph{Step 3}: Define $A_{3} = R_{2}Q_{2}$ and factor it as $A_{3} = Q_{3}R_{3}$.

 $\vdots$
 
\emph{Step $k+1$}: Define $A_{k + 1} = R_{k}Q_{k}$ (where $A_{k} = Q_{k}R_{k}$). 

Note that $A_{k + 1}$ is similar to $A_{k}$ (in fact, $A_{k+1} = R_{k}Q_{k} = (Q_{k}^{-1}A_{k})Q_{k}$), and hence each $A_{k}$ has the same eigenvalues as $A$. If the eigenvalues of $A$ are real and have distinct absolute values, the sequence of matrices $A_{1}, A_{2}, A_{3}, \dots$ converges to an upper triangular matrix with these eigenvalues on the main diagonal. 
\end{algorithm}

Compare the following example to Example \ref{QR-algortihm-2x2-025425}.

\begin{example}\label{ex:qrEig}
Use the $QR$ algorithm to approximate the eigenvalues of $A=\begin{bmatrix}1 & 1\\2 & 0\end{bmatrix}$.  

\begin{explanation}
    Here is a very simple implementation of the algorithm.  

    \begin{verbatim}
% Define A
A=[1 1;
    2 0];

% Set the desired number of iterations
iter=20;  

% Implement the algorithm. Note that we are over-writing the previous A with every iteration.
for i=1:iter
    [Q,R]=qr(A);
    A=R*Q
end    
    \end{verbatim}

\href{https://sagecell.sagemath.org/?z=eJwtjbEKwkAQRPuF_YdpAioWuZSGKw78gcQypFBuD7fwDjfn_5tDp3oMvJkOV0maBYEp-MXBjUzYM6Bfd2TqcJOK-hRE2dQkIn9eDzGUBK1i96olb0yN_dCPQLNSMah3l9b-BpfpPK_-bYdw_F8EP58mJsnxCzFCIso=&lang=octave&interacts=eJyLjgUAARUAuQ==}{Link to code}.   

Observe that the main diagonal entries are approaching the eigenvalues of $A$.
\end{explanation}
    
\end{example}

\end{document}