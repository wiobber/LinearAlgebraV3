\documentclass{ximera}
\input{../preamble.tex}

\title{Systems} \license{CC BY-NC-SA 4.0}
\begin{document}
\begin{abstract}
\end{abstract}
\maketitle

The templates in this section provide sample Octave code for solving systems of equations. You can access our code through the link at the bottom of each template.  Feel free to modify the code and experiment to learn more!  Alternatively, go to the \href{https://sagecell.sagemath.org/}{Sage Math Cell Webpage}, copy the code below into the cell, select OCTAVE as the language, and press EVALUATE.  

\begin{template}\label{temp:systems1}
Consider the system
\begin{equation}
\begin{array}{ccccccccc}
      x &- &y&&&&&= &0 \\
	 2x& -&2y&+&z&+&2w&=&4\\
     & &y&&&+&w&=&0\\
     & &&&2z&+&w&=&5
    \end{array}
    \end{equation}

 (You may remember this system from \href{https://ximera.osu.edu/linearalgebrav3/LinearAlgebraInteractiveIntro/SYS-0020/main}{Augmented Matrix Notation and Elementary Row Operations}.)  We will solve this system with Octave.   
\begin{verbatim}
% Define the coefficient matrix A
A = [1 -1 0 0;
     2 -2 1 2;
     0 1 0 1;
     0 0 2 1];

% Define vector b
b = [0;4;0;5];

% Solve the system 
x = A \ b
\end{verbatim}

\href{https://sagecell.sagemath.org/?z=eJxFjsEKwjAQRO-B_Ye59FjYBD0FDwH_wGP1YMIGA7aBNpT69yZIcG-PeczsgKvEtAjKSxCyxJhCkqVgfpY1HXCkHC6YNEYNBltSaGcwGmiYzowW6z9yVfSjMqmhb-wSSl7hSfnWyfZk2Z67dMvv_ffH9tmKzCB1VM3hDv8FLh8nkA==&lang=octave&interacts=eJyLjgUAARUAuQ==}{Link to code}.

\begin{warning}
    The system above has a unique solution.  Try manipulating the system so that it has infinitely many solutions or no solutions.  Run the code to see what happens.  You will observe the following:

\begin{itemize}
    \item If a system has infinitely many solutions, Octave will find one particular solution.
    \item If a system has no solutions, Octave will find an approximate solution using least-squares.  We will cover this method in \href{https://ximera.osu.edu/linearalgebrav3/LinearAlgebraInteractiveIntro/RTH-0030/main}{Least-Squares Approximation}.
\end{itemize}
\end{warning}

\end{template}

Due to the behavior of $A\setminus b$ function noted in the warning above, it is often more convenient to solve a system by referencing its reduced row echelon form.  When solving systems by hand, you utilized Gauss-Jordan elimination to find rref.  Having the reduced row echelon form available made it easy to write out the solution(s) or determine that the system is inconsistent.  We can find rref and the rank of a matrix using Octave as follows.

\begin{template}\label{temp:rref}
We will use the same system of equations as in Template \ref{temp:systems1}.
    \begin{verbatim}
% Define the augmented matrix 
% corresponding to the coefficient matrix A and vector b
A_b = [1 -1 0 0 0;
     2 -2 1 2 4;
     0 1 0 1 0;
     0 0 2 1 5];
     
% Find rref of A_b
rref(A_b)

% Find the rank of A_b
rank(A_b)
    \end{verbatim}

\href{https://sagecell.sagemath.org/?z=eJxFjsEKwjAMhu-FvsN_Gehh0A49iYfB8CVEpOvSWWSt1Co-vumkmBCS_-cjSYOBnA-EfCOY17xQyDRhMTn5D2xMiZ6PGCYfZuS4UjaSc956JivXw4QJb7I5JoxS9NcRR5w1Wg1V8iAFSnRoO2huu-oo6F_9DYXC7C_VkaLByfMBfsYhOvB6KYrY8LSVgoUJ91V8ASwdNw8=&lang=octave&interacts=eJyLjgUAARUAuQ==}{Link to code}.    
\end{template}


  
\end{document}