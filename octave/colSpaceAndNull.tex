\documentclass{ximera}
\input{../preamble.tex}

\title{Column Space and Null Space of a Matrix} \license{CC BY-NC-SA 4.0}
\begin{document}
\begin{abstract}
\end{abstract}
\maketitle
\section*{Column Space and Null Space of a Matrix}

The templates in this section provide code samples for finding the column space and the null space of a matrix.  Same procedures can be used for finding the image and the kernel of a linear transformation. You can access our Octave code through the link at the bottom of each template.  Feel free to modify the code and experiment to learn more!  Alternatively, go to the \href{https://sagecell.sagemath.org/}{Sage Math Cell Webpage}, copy the code below into the cell, select OCTAVE as the language, and press EVALUATE.  

\subsection*{Column Space (Image of a Linear Transformation)}

Recall that the column space of a matrix is the span of the columns of the matrix.  We have the following procedure for finding the column space.

\begin{procedure}[\ref{proc:colspace}]
Given a matrix $B$, a basis for $\mbox{col}(B)$ can be found as follows:
\begin{enumerate}
\item Find $\mbox{rref}(B)$ (or any row-echelon form $B'$ of $B$.)
\item Identify the pivot columns of $\mbox{rref}(B)$ (or $B'$).
\item The columns of $B$ corresponding to the pivot columns of $\mbox{rref}(B)$ (or $B'$) form a basis for $\mbox{col}(B)$.
\end{enumerate}
\end{procedure}

We will implement this procedure in Octave.

\begin{template}\label{temp:colSpace}
For this template we will use matrix $$A=\begin{bmatrix}2&-1&1&-4&1\\1&0&3&3&0\\-2&1&-1&5&2\\4&-1&7&2&1\end{bmatrix}$$
of Example \ref{ex:basiscolspace}.  We can use Octave to find the reduced row-echelon form of $A$, and identify the pivot columns visually.  We can also take advantage of the expanded version of the \emph{rref} command to identify the pivot columns automatically, as shown below.

\begin{verbatim}
% Define A
A = [2 -1 1 -4 1;
    1 0 3 3 0;
    -2 1 -1 5 2;
    4 -1 7 2 1];

% Compute rref of A
[rref_A, pivot_cols] = rref(A);

% the pivot column indices are...
pivot_cols

% Extract the pivot columns from A, using pivot column indices
colSpace_basis = A(:, pivot_cols)
\end{verbatim}

\href{https://sagecell.sagemath.org/?z=eJxtTkEKwjAQvAfyh7kULNhiqiIoHoL6Ao9SSq2JBmxTklR8vhssKOjuZWeGmdkEe6VNpyA5k9jiVCATEMgWEBvOQCMww5x2NuKsiLrAEsXILCJcgfiSGM4S7GzbD0HBOaVhdUw_xbuSU_TmYUPV2LsvqTCyE5mOvnBTbx2kD20H011Mozxqp_I85-xjfhsOz-DqJvwYPbSzLahu8Ka7_g3ljPCxrxtVnWtvPH0jJ-vvB9MX3FlRVA==&lang=octave&interacts=eJyLjgUAARUAuQ==}{Link to code}.

Compare these findings to our answer in Example \ref{ex:basiscolspace}.
\end{template}  

\subsection*{Null Space of a Matrix (Kernel of a Linear Transformation)}

Recall the following definition of the null space of a matrix.
\begin{definition}[\ref{def:nullspace}] Let $A$ be an $m\times n$ matrix.  The \dfn{null space} of $A$, denoted by $\mbox{null}(A)$, is the set of all vectors $\vec{x}$ in $\RR^n$ such that $A\vec{x}=\vec{0}$.
\end{definition}

We can use the reduced row-echelon form to find the null space, as shown in Example \ref{ex:dimnull}.  As you work with Octave, you may come across the \emph{null} command.  This command finds an orthonormal basis for the null space.  An orthonormal set of vectors consists of unit vectors that are pairwise orthogonal.  We demonstrate the \emph{null} function below.

\begin{template}
    We will use the matrix in Example \ref{ex:dimnull}.  Let 
$$A=\begin{bmatrix}2&-1&1&-4&1\\1&0&3&3&0\\-2&1&-1&5&2\\4&-1&7&2&1\end{bmatrix}$$
Find $\mbox{null}(A)$.

\begin{verbatim}
% Define A
A = [2 -1 1 -4 1;
    1 0 3 3 0;
    -2 1 -1 5 2;
    4 -1 7 2 1];

% Basis for null(A)
% Octave constructs an orthonormal basis, meaning that the basis elements are pairwise orthogonal unit vectors.
B=null(A)

% Verify that each vector in the basis is a unit vector
norm(B(:,1))
norm(B(:,2))

% Verify that the two vectors are orthogonal
dot(B(:,1),B(:,2))
\end{verbatim}

\href{https://sagecell.sagemath.org/?z=eJxtjkFLxDAUhO-B_Ie5LLTQyqauLCh7aPHuzYt4iPV1G2hfJEl38d-b2Kg9mITAm2S-mR0eaTBMaKVoccJLg1pBoT5APUiBuBT2uI17n-e6Se8Kd2iyckjjEVF_jYoUO3TaG4_BOvAyTUVbJvGpD_pC6C374JY-eGiGdWG0bN2sJ7wlV4WZNBs-I4w6xItWHTTRTJxcjvChjbsaT6v_bDnaFzYBF-qDdf5Giu70m53Sn8mZ4XOFku7H_BOGNxnx6C1HilSt6Ir7SpXlZmrKf7CJE672p8N30b9-UrzbkFFVZnwBTjpvrw==&lang=octave&interacts=eJyLjgUAARUAuQ==}{Link to code}.
\end{template}



\end{document}