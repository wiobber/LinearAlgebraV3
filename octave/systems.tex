\documentclass{ximera}
\input{../preamble.tex}

\title{Vector Operations} \license{CC BY-NC-SA 4.0}
\begin{document}
\begin{abstract}
\end{abstract}
\maketitle

The templates in this section provide sample code for solving systems of equations.  To use Octave, go to the \href{https://sagecell.sagemath.org/}{Sage Math Cell Webpage}, copy the code below into the cell, select OCTAVE as the language, and press EVALUATE.  We also provide a link to our code at the bottom of each template.

\begin{template}\label{temp:systems1}
Consider the system%
\footnote{You may remember this system from \href{https://ximera.osu.edu/linearalgebrav3/LinearAlgebraInteractiveIntro/SYS-0020/main}{Augmented Matrix Notation and Elementary Row Operations}.  } 
\begin{equation}
\begin{array}{ccccccccc}
      x &- &y&&&&&= &0 \\
	 2x& -&2y&+&z&+&2w&=&4\\
     & &y&&&+&w&=&0\\
     & &&&2z&+&w&=&5
    \end{array}
    \end{equation}

 You may remember this system from \href{https://ximera.osu.edu/linearalgebrav3/LinearAlgebraInteractiveIntro/SYS-0020/main}{Augmented Matrix Notation and Elementary Row Operations}.   
\begin{verbatim}
% Define the coefficient matrix A
A = [1 -1 0 0;
     2 -2 1 2;
     0 1 0 1;
     0 0 2 1];

% Define vector b
b = [0;4;0;5];

% Solve the system 
x = A \ b
\end{verbatim}

\href{https://sagecell.sagemath.org/?z=eJxFjsEKwjAQRO-B_Ye59FjYBD0FDwH_wGP1YMIGA7aBNpT69yZIcG-PeczsgKvEtAjKSxCyxJhCkqVgfpY1HXCkHC6YNEYNBltSaGcwGmiYzowW6z9yVfSjMqmhb-wSSl7hSfnWyfZk2Z67dMvv_ffH9tmKzCB1VM3hDv8FLh8nkA==&lang=octave&interacts=eJyLjgUAARUAuQ==}{Link to code}.

\end{template}

The system in Template \ref{temp:systems1} has a unique solution.  Try manipulating the system so that it has infinitely many solutions or no solutions.  Run the code to see what happens.
  
\end{document}