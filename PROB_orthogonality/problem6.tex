\documentclass{ximera}
\input{../preamble.tex}
\author{}
\license{Creative Commons 4.0 By-NC-SA}
%\outcome{Compute an antiderivative using basic formulas}
\begin{document}
\begin{exercise}

True or False?  If False, you should come up with a counterexample.  If True, can you give a proof?

 \begin{enumerate}
     \item If $A$ is symmetric, then $A$ has orthonormal columns.

 \begin{multipleChoice}
 \choice{True}
 \choice[correct]{False}
 \end{multipleChoice}

 \item The product of two orthogonal matrices is an orthogonal matrix.

 \begin{multipleChoice}
 \choice[correct]{True}
 \choice{False}
 \end{multipleChoice}

 \item The orthogonal complement of the row space of a matrix is the null space of that matrix.

 \begin{multipleChoice}
 \choice[correct]{True}
 \choice{False}
 \end{multipleChoice}

 \item For any proper subspace of $\RR^n$, it is possible to produce an orthogonal basis.

 \begin{multipleChoice}
 \choice[correct]{True}
 \choice{False}
 \end{multipleChoice}


 \end{enumerate}

 
\end{exercise}


\end{document}