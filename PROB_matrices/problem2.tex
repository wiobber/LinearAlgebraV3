\documentclass{ximera}
\input{../preamble.tex}
\author{}
\license{Creative Commons 4.0 By-NC-SA}
%\outcome{Compute an antiderivative using basic formulas}
\begin{document}
\begin{exercise}
Consider the system of equations:
$$\begin{matrix}
      x& -&3y&+&z&=&2\\
      -3x & -&4y&+&z&= &0\\
       & &2y&-&z&=&1
    \end{matrix}$$

Let $A$ be the coefficient matrix corresponding to this system, and let $\vec{b}=\begin{bmatrix}2\\0\\1\end{bmatrix}$.

Suppose that we find that  $(1, -2, -5)$ is a unique solution to this system.  What does this tell us?  Select ALL that apply.

\begin{selectAll}
 \choice[correct]{$A$ is invertable.}
 \choice[correct]{Vector $\vec{b}$  is in the span of the columns of $A$.}
 \choice{Columns of $A$ are linearly dependent.}
 \choice[correct]{Equation $A\vec{x}=\vec{b}$ has a unique solution.}
 \choice[correct]{Vector $\vec{b}$  can be written as a linear combination of the columns of $A$.}
 \choice{$A\vec{b}=\begin{bmatrix}1\\-2\\-5\end{bmatrix}$}
 \end{selectAll}
\end{exercise}

\end{document}