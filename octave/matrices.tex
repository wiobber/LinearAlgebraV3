\documentclass{ximera}
\input{../preamble.tex}

\title{Matrices} \license{CC BY-NC-SA 4.0}
\begin{document}
\begin{abstract}
\end{abstract}
\maketitle

The templates in this section provide sample code for basic matrix operations.  To use Octave, go to the \href{https://sagecell.sagemath.org/}{Sage Math Cell Webpage}, copy the code below into the cell, select OCTAVE as the language, and press EVALUATE.  We also provide a link to our code at the bottom of each template.

\begin{template}\label{temp:matrixOps}

\begin{verbatim}
% Define matrices A, B, and C
A = [1 -1 0 0;
     2 -2 1 2;
     0 1 0 1];
     
B = [2 3 -1 4;
     1 -1 2 -2;
     1 -2 0 3];   
     
C = [1 -2 1 4;
     2 -2 0 1;
     -2 1 1 0;
     1 3 -2 -1];     
     
% Find A+B
sum=A+B

% Find the transpose of A
% Copying and pasting A'  may not work because the prime comes through
% in a different font.
A_trans1=A'
% we can also use the transpose function
A_trans2=transpose(A)

% Find AC
product=A*C

% Find the inverse of C
inv_C=inv(C)

% Find the determinant of C
det_C=det(C)
\end{verbatim}

\href{https://sagecell.sagemath.org/?z=eJxdUEFuwyAQvFvyH-YSNWkTyZDcIg6Yqp-oogjZWEGqIcK4_X4XartOOKyY2ZmdhQ3eTWedQa9jsI0ZIPeo99CuhSoLCYFPhgNDhepcFkiH48DBwGdcIbXZZcZlUScbxzEZTzOdxyTviuBkPF7O2TR51RSZIk4PkZQx49xl_yuxlEUilmetpm3wYekp8q0ui2HsRb4sdLwZxKDdcPeDge8g6cnXzDAhX5Lux6DRDvpr8BhJ9GjpRtdE691i42JpbuVuFSXpN-_Bt2MThXxVT0tY923C3wrUInRVgupW7Z6ErYkm9NZpFycxMSSmSuJfcXxtIA==&lang=octave&interacts=eJyLjgUAARUAuQ==}{Link to code}.
\end{template}

\begin{template}\label{temp:id_matrix}
Sometimes it is useful to generate an identity matrix.  To generate a $5\times 5$ identity matrix, use the code below.
\begin{verbatim}
% n by n identity matrix
I_5=eye(5)
\end{verbatim}
\end{template}

\begin{template}\label{temp:matrixRowColEntry}
    We can access individual rows, columns, and entries of a matrix.
\begin{verbatim}
% Define a matrix 
A=[2 -1 3 -1; 1 0 2 1; 1 -1 1 -2]

% Third row of A
row_3=A(3,:)

% Second column of A
col_2=A(:,2)

% Second row third entry of A
a_23=A(2,3)   
\end{verbatim}
\href{https://sagecell.sagemath.org/?z=eJxVjEEKwjAURPeB3GE2BQsR7M-u0kXAG9SdSAhtigGbQIhYb-9v7cbNMDM8XoWLn0L0cJhdyWGBFKa7EY4NNMcZDU4gbIU_DrpLIUWF6yPkETm9kSYYKbhZ3ZmDVm39I3o_pDhiSM_XHHeKhyWmWkX_1Coqm9LHkj877iytTlK6_gIqSCuZ&lang=octave&interacts=eJyLjgUAARUAuQ==}{Link to code}.
\end{template}

\begin{example}\label{ex:gauss_octave}
Use Octave code to show one step of Gaussian elimination by subtracting half of the first row from the second row of the given matrix $A$.
\begin{verbatim}
% Define matrix A
A=[2 -1 3 -1; 1 0 2 1; 1 -1 1 -2] 

% Replace the second row of A with second row minus half the first row.  
A(2,:)=-1/2*A(1,:)+A(2,:)  
\end{verbatim}

\href{https://sagecell.sagemath.org/?z=eJxNjMsKwjAURPeB_MNsCvVRNXGndHHBL3ArLkK9IYE2kSRSP99YN26Gw5lhGlzY-sCYTEn-DZKC-ptGp3CscYbCARoLVFdD3yGFFA2u_BzNwCiOkXmI4YEUZ0QLwuyL-5eTD68MZ0a7zK1PuXyLHeoZtXp7WvWd2us1tary5qc-770qUw==&lang=octave&interacts=eJyLjgUAARUAuQ==}{Link to code}.
\end{example}

\end{document}