\documentclass{ximera}
\input{../preamble.tex}

\title{Challenge Problems for Ch 8} \license{CC BY-NC-SA 4.0}

\begin{document}

\begin{abstract}
\end{abstract}
\maketitle

\section*{Challenge Problems for Chapter 8}

\begin{problem}\label{prob:8_1_13}
If $U$ is a subspace of $\RR^n$, show that $\left(U^{\perp}\right)^\perp = U$. 
\begin{hint}
Show that $U \subseteq \left(U^{\perp}\right)^\perp$, then use Theorem~\ref{th:023783c} twice.
\end{hint}
\end{problem}


\begin{problem}\label{prob:8_1_14}
If $W$ is a subspace of $\RR^n$, show how to find an $n \times n$ matrix $A$ such that $W = \{\vec{x} \mid A\vec{x} = \vec{0}\}$. 

%\begin{hint}
%Use Problem~\ref{prob:8_1_13}.
%\end{hint}

%\begin{hint}
%Let $\{\vec{y}_{1}, \vec{y}_{2}, \dots, \vec{y}_{m}\}$ be a basis of $W^\perp$, and let $A$ be the $n \times n$ matrix with rows $\vec{y}^T_1, \vec{y}^T_2, \dots, \vec{y}^T_m, 0, \dots, 0$. Then $A\vec{x} = \vec{0}$ if and only if $\vec{y}_{i} \dotp \vec{x} = 0$ for each $i = 1, 2, \dots, m$; if and only if $\vec{x}$ is in $W^{\perp \perp} = U$.
%\end{hint}
\end{problem}

\begin{problem}\label{prob:8_1_16}
If $U$ and $W$ are subspaces, define $U+W$ to be the set of all possible sums of elements of $U$ with elements of $W$.

\begin{enumerate} 
\item Is $U+W$ a subspace?
Show that $(U + W)^\perp = U^\perp \cap W^\perp$. 
\item Illustrate this result with a diagram in $\RR^3$ where $U$ and $W$ are two distinct lines through the origin.  What does $U+W$ look like?  What do $U^{\perp}$ and $W^{\perp}$ look like?
\end{enumerate}
\end{problem}

\begin{problem}\label{prob:8_1_17.2}
A square matrix $E$ satisfying $E^2=E=E^T$ is called a \emph{projection matrix}.  If $E$ is a projection matrix, show that $I - E$ is also a projection matrix.
\end{problem}


\begin{problem}\label{prob:8_1_17.1}
\item Let $E$ be an $n \times n$ matrix, and let $W = \{\vec{x} E \mid \vec{x} \mbox{ in } \RR^n\}$. Show that the following are equivalent.


\begin{enumerate}
\item $E^{2} = E = E^{T}$ ($E$ is a \dfn{projection matrix}).

\item $(\vec{x} - \vec{x}E) \dotp (\vec{y}E) = 0$ for all $\vec{x}$ and $\vec{y}$ in $\RR^n$.

\item $\mbox{proj}_W(\vec{x}) = \vec{x}E$ for all $\vec{x}$ in $\RR^n$.
\begin{hint}
For (ii) implies (iii): Write $\vec{x} = \vec{x}E + (\vec{x} - \vec{x}E)$ and use the uniqueness argument preceding the definition of $\mbox{proj}_W(\vec{x})$. For (iii) implies (ii): $\vec{x} - \vec{x}E$ is in $W^\perp$ for all $\vec{x}$ in $\RR^n$.
\end{hint}
\end{enumerate}
\end{problem}



\begin{problem}\label{prob:8_1_17.3}
If $EF = 0 = FE$ and $E$ and $F$ are projection matrices, show that $E + F$ is also a projection matrix.

\end{problem}

\begin{problem}\label{prob:8_1_17.4}
If $A$ is $m \times n$ and $AA^{T}$ is invertible, show that $E = A^{T}(AA^{T})^{-1}A$ is a projection matrix.

Click the arrow to see answer.

\begin{expandable}
    $E^T = A^T[(AA^T)^-1]^T(A^T)^T  = A^T[(AA^T)^T]^{-1}A = A^T[AA^T]^{-1}A = E$
\end{expandable}

\end{problem}




\begin{problem}\label{prob:ortho11}
Consider $A = \begin{bmatrix}
0 & a & 0 \\
a & 0 & c \\
0 & c & 0
\end{bmatrix}$
 where one of $a, c \neq 0$. Show that the characteristic polynomial (see Definition \ref{def:char_poly_complex}) is given by $c_{A}(z) = z(z - k)(z + k)$, where $k = \sqrt{a^2 + c^2}$, and find an orthogonal matrix $Q$ such that $Q^{-1}AQ$ is diagonal.

Click the arrow to see the answer.
\begin{expandable}
$Q = \frac{1}{\sqrt{2}k}\begin{bmatrix}
c\sqrt{2} & a & a \\
0 & k & -k \\
-a\sqrt{2} & c & c
\end{bmatrix}$
\end{expandable}
\end{problem}

\begin{problem}\label{prob:ortho13}
Given $A = \begin{bmatrix}
b & a \\
a & b
\end{bmatrix}$, show that the characteristic polynomial (see Definition \ref{def:char_poly_complex}) is given by $c_{A}(z) = (z - a - b)(z + a - b)$, and find an orthogonal matrix $Q$ such that $Q^{-1}AQ$ is diagonal.
\end{problem}


\section*{Bibliography}

Some of these problems come from Section 7.4 of Ken Kuttler's \href{https://open.umn.edu/opentextbooks/textbooks/a-first-course-in-linear-algebra-2017}{\it A First Course in Linear Algebra}. (CC-BY)

Ken Kuttler, {\it  A First Course in Linear Algebra}, Lyryx 2017, Open Edition, pp. 433--438.  

Other problems come from the second part of Section 8.1 of Keith Nicholson's \href{https://open.umn.edu/opentextbooks/textbooks/linear-algebra-with-applications}{\it Linear Algebra with Applications}. (CC-BY-NC-SA)

W. Keith Nicholson, {\it Linear Algebra with Applications}, Lyryx 2018, Open Edition, p. 422--423 

\end{document}